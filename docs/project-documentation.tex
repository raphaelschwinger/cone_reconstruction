\documentclass{article}

% Language setting
% Replace `english' with e.g. `spanish' to change the document language
\usepackage[english]{babel}

% Set page size and margins
% Replace `letterpaper' with`a4paper' for UK/EU standard size
\usepackage[letterpaper,top=2cm,bottom=2cm,left=3cm,right=3cm,marginparwidth=1.75cm]{geometry}

% Useful packages
\usepackage{amsmath}
\usepackage{graphicx}
\usepackage[colorlinks=true, allcolors=blue]{hyperref}

\title{Masterproject Deep Learning und Autonomous Racing
}
\author{Rakibuzzaman Mahmud
Raphael Schwinger}

\begin{document}
\maketitle


\section{Project Overview}

- The master project "Ground Truth Generation"  is a part of the "Rosyard" project. 

- The SLAM algorithm will use the results of the master project to optimize the car by getting an accurate ground truth of the track and the car's position during a test race.

- This task of ground truth generation for the SLAM algorithm is divided into two subtasks. First, a ground truth of the race track has to be generated.  Second, the position of the car has to be recorded during a race. 


- A race track of Rosyard usually roughly covers 50 x 150 meters and is marked by two rows of cones placed parallel to each other. The cones are placed at intervals of 5 meters lengthwise to the direction of travel, the width of the track is about 3 meters, and a cone has a height of 32 cm. 

- In order to define the ground truth of the race track, it is therefore sufficient to determine the positions of these cones. We will use  3D scene reconstruction using images/videos of the race track. 



\section{Introduction}

\subsection{introduce Raceyard project}
\subsection{introduce problem, why does car needs to be tracked}
\subsection{discussion on what methods could be used}
\subsection{what did the premilary group}


\section{Reconstruction of the racetrack}

\subsection{Structure from motion}
\subsection{Blender, why}
\subsection{ transforming output to "real world" with affine transformation}


\newpage

\section{Tracking the racecar}

\subsection{ Picture information from blender}
\subsection{Tracking with openCV}

Comparison of different Object detection algorithm from openCV and how we choose CSRT algorithm as the best one.
\subsection{exact tracking? }
\subsection{results}





\section{Miscs}

\subsection{hassle with python dependencies
Docker dev Container}
\subsection{Pypangolin version error fix description}

\subsection{Dev Container}
Our implementation depends on various software packages in different versions, that's why we worke on a Docker dev container. Descirption of how we used dev container to manage our virtualbox.



\end{document}